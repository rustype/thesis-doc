%!TEX root = ../template.tex
%%%%%%%%%%%%%%%%%%%%%%%%%%%%%%%%%%%%%%%%%%%%%%%%%%%%%%%%%%%%%%%%%%%%
%% abstrac-pt.tex
%% NOVA thesis document file
%%
%% Abstract in Portuguese
%%%%%%%%%%%%%%%%%%%%%%%%%%%%%%%%%%%%%%%%%%%%%%%%%%%%%%%%%%%%%%%%%%%%

\typeout{NT FILE abstract-pt.tex}

À medida que as nossas vidas estão cada vez mais dependentes de software,
os erros do mesmo têm o potencial de causar problemas significativos.
Prevenir estes erros torna-se uma tarefa prioritária durante o desenvolvimento de sistemas confiáveis.
Erradicar erros por completo é impossível, mas é possível eliminar certos conjuntos.

Rust é uma linguagem de programação de sistemas que, por desenho, endereça erros de gestão de memória.
Para o conseguir, a linguagem inclui no sistema de tipos informação sobre o tempo de vida dos objetos,
permitindo assim que o compilador conheça a utilização dos mesmos e detecte erros de utilização de memória.
Apesar da prevenção de erros de memória ter um papel importante na segurança de software,
existem ainda outras categorias de erros em Rust,
como o uso incorrecto de interfaces de programação, em que o programador não respeita as restrições impostas pela mesma, o que resulta numa falha do programa.

\emph{Typestates} elevam o conceito de estado para o sistema de tipos,
permitindo a aplicação das restrições da interface durante a fase de compilação.
Este conceito permite assim aliviar o programador da responsabilidade que é conceptualizar e manter o estado do programa em mente durante o desenvolvimento, prevenindo o mau uso das interfaces.
Apesar de Rust não suportar \emph{typestates} de uma forma natural,
o sistema de tipos permite expressar e validar \emph{typestates}.

O objetivo desta tese é aproximar os \emph{typestates} do Rust em produção,
desenvolvendo uma ferramenta que permite aos programadores tirar partido dos \emph{typestates}.
A ferramenta toma a forma de uma DSL embebida em Rust, apoiada por macros,
permitindo que a abordagem não se desvie da sintaxe de Rust e seja fácil de usar.
O programador especifica \emph{typestates} pelo uso de anotações no código.
Apartir da especificação, extrai-se uma máquina de estados que é depois verificada por erros comuns de \emph{typestates},
garantindo propriedades extra em comparação com \emph{typestates} comuns.
A ferramenta tira partido do sistema de tipos de Rust para garantir que o código está em acordo com os \emph{typestates}.

% Palavras-chave do resumo em Português
\begin{keywords}
Tipos comportamentais, \emph{typestates}, meta-programação, macros, Rust
\end{keywords}
% to add an extra black line

%!TEX root = ../template.tex
%%%%%%%%%%%%%%%%%%%%%%%%%%%%%%%%%%%%%%%%%%%%%%%%%%%%%%%%%%%%%%%%%%%%
%% chapter6.tex
%% NOVA thesis document file
%%
%% Chapter with lots of dummy text
%%%%%%%%%%%%%%%%%%%%%%%%%%%%%%%%%%%%%%%%%%%%%%%%%%%%%%%%%%%%%%%%%%%%

\typeout{NT FILE chapter6.tex}

\chapter{Conclusions \& Future Work}\label{cha:conclusions}

\section{Summary}

In this thesis a new tool is presented for the development of typestates in Rust,
a language without first-class support for such concept.
While it is possible to model typestates in plain Rust, it is cumbersome and error-prone;
my work takes advantage of Rust's advanced type and macro systems
to provide an embedded DSL which simplifies the development of typestates in Rust.

To the best of my knowledge, while similar works have been developed in other languages,
the present work is the first one to leverage macros to automate
and provide an elegant DSL along with extra features over Rust's existing typestate capabilities.

While the DSL is a single contribution, its parts constitute several points of improvement over the current \emph{status quo}:
\begin{description}
    \item[Tool independent DSL.] The DSL requires no dependencies external to Rust itself,
    a user is able to start working with my work by simply adding a dependency to their project.
    \item[Verification tool.] The macro executes several verifications over the declared typestate (discussed in \autoref{sec:validation}).
    \item[Visualization tool.] The macro behind the DSL also works as a visualization tool for the developed typestates (discussed in \autoref{sec:automata-visualization}),
    it provides three formats, one of which can be embedded in the client's API, avoiding documentation rot.
\end{description}

The capabilities provided by the macro are useful for large systems and provide extra assurance during development,
any system required to \quotes{get it right} can take advantage of them,
reducing runtime state checks and easing the development process of large stateful systems.

\section{Future Work}

The current version of the macro suffers from some limitations:
it does not support generics and by consequence, state machine nesting.
Furthermore, the present work lacks formalization, relying on the Rust compiler to provide part of the formal guarantees;
formalizing the present work represents a great step towards providing more guarantees that can be used in the \quotes{real-world}.

While the previous points aim to provide extra features, the macro would also benefit from some work in its internals.
Everything works as expected, however the architecture and code organization could be made cleaner by further dividing each visitor.
Currently, some visitors perform mutations to the tree at the same time they extract relevant information;
this process can be split into mutation and information extraction, easing code readability.
While I have no concrete proof of the performance impact of such change,
I believe that allowing the macro to be more modular and simpler to understand, trumps other concerns.

Along with the visitor changes, the underlying graph structures could also be modified,
using a single graph structure as the result of the information extraction process,
the graph could then be \quotes{reduced} into several specialized use cases, such as state machine verification and visualization.

Finally, planning for the development of such work in the future ---
I believe that starting with the re-architecture of the visitors followed by the graph would yield the best results,
allowing changes in the future to be integrated in a simple manner;
leaving the formal work and new features for a more stable and robust version of the macro.

% \begin{figure}
%     \centering
%     \begin{tikzpicture}
%         \tikzstyle{code} = [fill=white, font=\tiny, align=left]
%         \tikzstyle{phase} = [below, font=\small\itshape]

%         \def\xSpec{0}
%         \def\xAST{4}
%         \def\xFSM{8}
%         \def\xCode{12}
%         \def\yLabel{-1.75}

%         \node[code] (code) at (\xSpec,0) {\mintinline{rust}{#[typestate]}\\\mintinline{rust}{// ...}};

%         \begin{scope}[shift={(\xAST, 0.75)}]
%             \tikzstyle{n}=[circle, draw=blue!70, fill=blue!20]
%             \node[n] (root) at (0, 0) {};
%             \node[n] (l1) at (-0.35, -0.75) {};
%             \node[n] (l2) at (0.35, -0.75) {};
%             \node[n] (l11) at (-0.7, -1.5) {};
%             \draw[-] (root) -- (l1);
%             \draw[-] (root) -- (l2);
%             \draw[-] (l1) -- (l11);
%         \end{scope}

%         \begin{scope}[shift={(\xFSM, 0.75)}]
%             \tikzstyle{n}=[circle, draw=blue!70, fill=blue!20]
%             \tikzstyle{f}=[circle, draw=red!70, fill=red!20]
%             \tikzstyle{s}=[circle, draw=green!70, fill=green!20]
%             \node[s] (root) at (0, 0) {};
%             \node[n] (l1) at (-0.5, -0.75) {};
%             \node[n] (l2) at (0.5, -0.75) {};
%             \node[f] (l11) at (0, -1.5) {};
%             \draw[->] (root) -- (l1);
%             \draw[->] (root) -- (l2);
%             \draw[->] (l1) -- (l11);
%             \draw[->] (l2) -- (l1);
%         \end{scope}

%         \node[code] (rust-code) at (\xCode,0) {\mintinline{rust}{struct S { ... }}\\\mintinline{rust}{trait SOps { ... }}\\\mintinline{rust}{// ...}};

%         % \draw[->, thick] (code) -> (2.5, 0);
%         % \draw[->, thick] (4.5, 0) -> (5, 0);
%         % \draw[->, thick] (7, 0) -> (rust-code);

%         \node[align=center] (label-1) at (\xSpec, \yLabel) {Typestate\\Specification};
%         \node[align=center] (label-2) at (\xAST, \yLabel) {AST};
%         \node[align=center] (label-3) at (\xFSM, \yLabel) {State\\Machine};
%         \node[align=center] (label-4) at (\xCode, \yLabel) {Rust\\Code};

%         \draw[->, thick] (label-1) -- node[phase] {Parse} (label-2);
%         \draw[->, thick] (label-2) -- node[phase] {Convert} (label-3);
%         \draw[->, thick] (label-3) -- node[phase] {Check: Ok} (label-4);
%         \draw[->, thick] (label-3) edge[in=-35, out=-145] node[phase] {Check: Error} (label-1);

%     \end{tikzpicture}
%     \caption{
%         From DSL specification to Rust code.
%         First the DSL is parsed, then converted to a state machine and the properties checked
%         (in the case some property is not respected, an error is issued).
%         Once the properties are validated, the Rust code is generated.
%     }
%     \label{fig:dsl-processing}
% \end{figure}

\begin{figure}
    \centering
    \begin{tikzpicture}
        \tikzstyle{n} = [draw, align=center]
        \node[n] (node-1) {Typestate\\Specification};
        \node[n, right=of node-1, xshift=1cm] (node-2) {AST};
        \node[n, below=of node-2, yshift=-.5cm] (node-3) {Intermediate\\Graph};
        \node[n, below=of node-3, yshift=-.5cm] (node-4) {State\\Machine};
        \node[n, right=of node-4, xshift=1.5cm] (node-5) {Rust\\Code};
        \node[n, right=of node-3, xshift=1.5cm] (node-6) {Visualization(s)};

        \draw[->, thick] (node-1) -- node[above] {Parse} (node-2);
        \draw[->, thick] (node-2) -- node[right] {Extract} (node-3);
        \draw[->, thick] (node-3) -- node[right] {Convert} (node-4);
        \draw[->, thick] (node-4) -- node[above] {Check: Ok} (node-5);
        \draw[->, thick, dashed] (node-3) -- node[above] {Generate} (node-6);
        \draw[->, thick] (node-4) -| node[above right] {Check: Error} (node-1);
    \end{tikzpicture}
    \caption{
        From DSL specification to Rust code.
        First the DSL is parsed, an intermediate graph representing the automaton in more general terms is extracted from the AST,
        from the graph the macro will convert  the user can generate visualizations (for debugging or documentation), this step is optional.
    }
    \label{fig:dsl-processing}
\end{figure}